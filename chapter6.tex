%%%%
\chapter{まとめと課題} \label{conclusion}

\section{まとめ}
本研究では、物理系を定義できるシミュレータ \simname を提案した。学習者は\simname を用いることで、物理系を定義することができる。学習者は、動作例と比較することで、自身が定義した物理系の運動と現実の運動との対応を視覚的に確認することができる。

また、実装の方針も説明した。 lively.next と SymPy 、Pyodide を用い、学習者はブラウザのみで \simname が実行できるようにする。これにより、ソフトウェアをインストールする手間がなく、手軽に利用できる。

\section{今後の課題}

\subsection{実装}

今後は、まず実装を進めることが課題である。未実装の内容は大きく分けると、
\begin{enumerate}
  \item lively.next で方程式を入力するインターフェースを作る
  \item SymPy での計算結果をもとに lively.next 描画する
  \item SymPy での次元の不正な立式を防ぐ
\end{enumerate}
である。このうち、3) については Python 上で実装すればよく、比較的容易と考えている。一方 1) 2) に関しては、フロントエンドに lively.next を用いることが最適かどうかを先に検討する必要がある。

\subsection{評価}

実装が完成したら、\simname の教育効果を評価する必要がある。高校生を、\simname を用いるグループ・PhET を用いるグループ・座学のグループに分け、pretest と posttest を行う。その結果を Hake~\cite{hake_1998}が導入した Normalized Gain を用いて比較することで、\simname の教育効果を測定できると考える。

また、\simname の利点は、現実の運動を確認できるだけでなく、学習者が定義した方程式と現実の物理法則に従う物体の運動の対応を理解できるという点である。そのため、各グループでディスカッションを行い、その内容からどれだけ現実の物理法則を正しく理解できているかを評価する。


\clearpage
\section{発展のアイデア}

\simname を発展させるアイデアがいくつかあるので紹介する。

\subsection{他分野への対応}

現在の \simname の想定では、力学以外の分野に対応することが難しい。物体を追加した際の次元付き変数や観測ペインの可視化が力学を前提とした設計となっているためである。一方高等学校で扱う物理には、音や光などの波動分野、回路や電場・磁場などの電磁気分野なども存在する。そのため、これらの分野にも対応できるようにしたい。

\subsection{方程式計算の強化}

現在、\simname 上に方程式を入力する際、その方程式の導出は学習者が行う必要がある。しかし、移項した際の符号の変え忘れや係数の間違いなど、\simname では検出できないミスが存在する。この作業を \simname でサポートしたい。

\subsubsection*{求解の自動化}
現在想定している \simname の実装では、物体に紐付けられている変数は他の変数で明示的に表されている必要がある。すなわち、$v_{Ax} = \sqrt{2gh}$ という表記は正しいが、$mgh = \dfrac{1}{2}mv_{Ax}^2$ という表記は受け付けない。このようなとき、SymPy を用いて $v_{Ax}$ を求めることができるため、計算ミスを防ぐことができる。一方、これを全て自動化してしまうと、学習者自身の方程式を変形する能力が成長しない。そのため、自動で求解するかどうかを切り替えられるようにする必要がある。

\subsubsection*{基本的な公式(運動方程式、力学的エネルギー保存則等)の提供}
物理学では、頻出する公式の数はある程度限られている。基本的な公式を提供し、各値に変数を割り当てるだけにする。これと先述した自動求解を組み合わせることで、適切な公式を選び適切に変数を割り当てるだけで物理系が作成できる。また、公式を解説付きで一覧にしたり、公式の検索機能をつけることで、公式を覚えきれてない初学者に対してもサポートができる。
