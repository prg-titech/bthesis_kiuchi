% TODO: 「物理教育のために物理実験シミュレータが使われている」という文で始めるのがよい

高等学校での物理学の授業では、実験が重要である。しかし実際は、生徒全員が実験を経験しているわけではない。理由としては、実験用の装置の準備や測定が難しいことや、実験を行うのに時間を要することが考えられる。そこで実験の代替として近年利用されているのが、物理実験のシミュレータである。シミュレータを用いることで、実験と同様の学習効果を得ることができる。

しかし既存のシミュレータでは、学習者は可視化されている運動とその背景に存在する数式の間の関係を理解することは容易ではない。既存のシミュレータでは、教育者が物理法則から現象を数式で記述し、シミュレーションに変換する。学習者はシミュレーションを見ることで現実の物理現象を確認することができるが、それと物理法則との関係性は授業などで学ぶのみである。
% TODO: 具体的な弊害

本研究は、学習者に物理系を定義させるシミュレータ \simnamealt~(\simname) を提案する。学習者は \simname で系内の物体をそのパラメータとともに定義し、その物体の運動を表す方程式を立式する。シミュレーションを実行すると、定義した物理系に基づいて数値計算がなされ、物体の運動が可視化される。これにより、現実の物理現象と物理法則の間の対応を学習者自らの経験を通して理解することができると考えられる。この際学習者は、\simname に実装された動作例を参考にすることで、定義した物理系と現実の運動を比較することができたり、次元の異なる物理量の和が存在する不正な方程式を立式すると警告されるなど、正しい物理系を作成するための補助を受ける。

% TODO: この論文で何が示されるのか(たとえば実装のアイディア)を述べましょう。
