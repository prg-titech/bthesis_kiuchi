高等学校での物理学の授業では、実験は重要である。しかし実際は、生徒全員が実験を経験しているわけではない~\cite{2015KJ00010038066}。理由としては、実験用の装置の準備や測定が難しいことや、実験を行うのに時間を要することが考えられる。そこで実験の代替として近年利用されているのが、物理実験のシミュレータである。シミュレータを用いることで、実験と同様の学習効果を得ることができる~\cite{ajredini_real_2014}。

しかし、既存のシミュレータでは不十分な点が存在する。理論を学習する上では物理法則やそれを前提とした運動を方程式を通して学習する。一方、既存のシミュレータでは速度や位置のような物理量を数値でしか確認できず、描画されている物理系がどのような方程式によって表現されているのかわからない。

そこで本研究では、物理系を定義できるシミュレータである \simname~(\simnamealt) を提案する。学習者は \simname で系内の物体をそのパラメータとともに定義し、その物体の運動を表す方程式を立式する。この際学習者は、次元の異なる物理量の和が存在する不正な方程式を立式すると警告されるなど、正しい物理系を作成するための補助を受ける。シミュレーションを実行すると、定義した物理系に基づいて数値計算がなされ、物体の運動が可視化される。さらに、 \simname には現実の運動を正しく表現した動作例が存在し、学習者が参考にすることができる。

\simname の利点は、現実の運動を確認できるだけでなく、学習者が定義した方程式と現実の物理法則に従う物体の運動の対応を理解できるという点である。既存のシミュレータはツール側が物理系を全て提供し、学習者はパラメータを指定するだけである。一方 \simname では、学習者は動作例と比較しながら物理系を定義することで、現実の運動がどのような方程式で表されるものなのか理解できる。