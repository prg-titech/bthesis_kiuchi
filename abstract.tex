高等学校での物理学の授業では、実験は重要である。しかし実際は、生徒全員が実験を経験しているわけではない~\cite{2015KJ00010038066}。理由としては、実験用の装置の準備や測定が難しいことや、実験を行うのに時間を要することが考えられる。そこで実験の代替として近年利用されているのが、物理実験のシミュレータである。シミュレータを用いることで、実験と同様の学習効果を得ることができる~\cite{ajredini_real_2014}。

しかし、既存のシミュレータでは不十分な点が存在する。既存のシミュレータは物理法則から現象を数式で記述し、シミュレーションに変換するまでを教育者が行なっている。学習者はシミュレーションを見ることで現実の物理現象を確認することができるが、それと物理法則との関係性は授業などで学ぶのみである。

そこで本研究では、学習者に物理系を定義させるシミュレータである \simname~(\simnamealt) を提案する。学習者は \simname で系内の物体をそのパラメータとともに定義し、その物体の運動を表す方程式を立式する。この際学習者は、次元の異なる物理量の和が存在する不正な方程式を立式すると警告されるなど、正しい物理系を作成するための補助を受ける。シミュレーションを実行すると、定義した物理系に基づいて数値計算がなされ、物体の運動が可視化される。さらに、 \simname には現実の運動を正しく表現した動作例が存在し、学習者が参考にすることができる。

\simname では、物理法則から現象を数式で記述し、シミュレーションに変換するまでの工程も学習者が行う。これにより、従来は授業などで教わる必要があった現実の物理現象と物理法則の間の対応を学習者自らの経験を通して理解することができる。
