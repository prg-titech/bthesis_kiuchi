\documentclass[11pt, a4paper, oneside, twocolumn]{jsarticle}
\usepackage{url}
\usepackage[dvipdfmx]{graphicx}
\usepackage{geometry}
\geometry{top=1cm,  bottom=2cm, left=1.5cm, right=1.5cm}
\pagestyle{empty}
\usepackage{amsmath}

\newcommand{\simname}{SimSym(仮名)}
\newcommand{\simnamealt}{Simulation with Symbols}

\title{物理学の学習のためのプログラマブルな
\\シミュレータと環境の提案}

\author{東京工業大学 情報理工学院 数理・計算科学系\\18B04657 木内康介\\指導教員 増原英彦教授}

\date{}

\begin{document}
\maketitle

\section{はじめに} \label{intro}
高等学校における物理学の授業において、生徒による実験は必要不可欠である。しかし実際は、実験の実施は完全ではない。林らの調査~\cite{2015KJ00010038066}によると、力学分野における最も基本的な「運動の法則」に関する実験の経験は、2014年の調査時点で60\%程度にしか満たない。また、非常によく取り上げられる題材である「モンキーハンティング」に関しては10\%に満たない。考えられる理由としては、モンキーハンティングを実験するためには大掛かりな装置と空間が必要になることや、測定が難しいことなどがある。

そこで代替として考えられるのが、シミュレーションの利用である。簡単に実験ができない内容であったとしても、シミュレーションを用いれば誰でも実験と同様な学習効果を得ることができる。実際 Ajredini~\cite{ajredini_real_2014}は、実際の実験とシミュレーションで得られる知識に大きな差は無いと結論づけている。

% しかし、既存のシミュレータでは不十分な点も存在する。一般的な物理学の学習用のシミュレータでは、設定されたシチュエーションにおける物体の挙動を観測することはできるが、シチュエーションそのものを大きく変化させることはできない。即ち、「この物体の座標を変更するとどう動くか」「この物体の初速度を変更するとどう動くか」「重力加速度の値を変更するとどう動くか」などといった疑問全てに対するシミュレーションは提供できない。更に、シミュレーションはその特性上数値計算をベースに実行される。一方、大学入試などにおける物理の問題は文字式の計算をベースにしている。そのため、自分が計算した結果の文字式と自分が想定している動きが一致しているかを確認することができない。

しかし、既存のシミュレータには不十分な点が存在する。それは、シミュレーションは数値計算をベースに実行されることである。一方、大学入試などにおける物理の問題は文字式の計算をベースにしている事がほとんどである。そのため、既存のシミュレータでは自分が計算した結果の文字式と自分が想定している動きが一致しているか直接確認することはできない。

そこで本論文では、文字式をベースとしたシミュレーションが可能なシミュレータである「\simname」を提案する。文字式をそのまま再現することができるシミュレータを用いれば、自分の計算結果を「実行」し、自分が想定している動きを正しく表せているかを確認できる。それにより、文字式と動きの対応を学習することができ、文字式に対するある種の直感的な理解を与えることができる。結果として、自分が計算した文字式が妥当かを判断したり、選択肢中に存在するありえない択を排除したりすることができるようになると考える。

\section{提案する内容}
\simname (\simnamealt) は、文字式をベースとしたシミュレーションを行うことができるシミュレータである。ユーザは、\simname 上で文字式による計算を実行でき(\ref{step1})、文字式を物体の動きに紐付けることで(\ref{step2})、文字式に基づいたシミュレーションを実行することができる(\ref{step3})。

\subsection{文字式による計算} \label{step1}
\simname では、文字式による計算をサポートしている。即ち、変数を用いて立式し、移項や代入をし、問われた値を求める、という一連の操作を \simname 上で実行することができる。

\begin{figure}[h]
\centering
後で図を作って挿入
\caption{\simname 上で文字式を計算する例}
\end{figure}

\subsection{文字式の紐付け} \label{step2}
\ref{step1} で作成した文字式を、シミュレータに存在する物体の動きに紐付けることができる。これにより、あらゆる動きをユーザが自由に設定することができる。

\begin{figure}[h]
\centering
後で図を作って挿入
\caption{\simname 上の物体の動きに文字式を紐付ける例}
\end{figure}

\subsection{シミュレーションの実行} \label{step3}
\ref{step2} で紐付けた文字式に対し数値を代入することで、シミュレーションを実行する。ユーザは、自身が計算した文字式が意図した動作に対応しているかをシミュレーションを通して確認することができる。

\section{実装}
実装は、フロントエンドに lively.next\footnote{\url{https://lively-next.org}} を、文字式の計算に SymPy\footnote{\url{https://www.sympy.org/en/index.html}} を用いた。SymPy は Python のライブラリであるが、WebAssembly で実装された CPython 処理系の Pyodide\footnote{\url{https://pyodide.org/en/stable/}} を用いることでブラウザ上で完結させた。また、数式の表示には KaTeX\footnote{\url{https://katex.org}} を用いた。

\section{評価手法の提案}
\simname の教育効果の評価については、Hake~\cite{hake_1998}が導入した normalized gain を用いる方法を提案する。満点を100とする pre-test と post-test の平均点をそれぞれ $\langle \textrm{pre\\-test} \rangle$, $\langle \textrm{post\\-test} \rangle$ とすると、normalized gain $\langle g \rangle$ は次のように定義される:
$$ \langle g \rangle = \dfrac{\langle \textrm{post\\-test} \rangle - \langle \textrm{pre\\-test} \rangle}{100 - \langle \textrm{pre\\-test} \rangle} $$
例えば、平均の得点率が $20\%$ から $60\%$ になった場合と $60\%$ から $80\%$ になった場合では、どちらも $\langle g \rangle = 0.5$ となる。

具体的には、高校生数十人を対象とした調査を行う。対象を
\begin{itemize}
  \item {シミュレータを利用しないグループ}
  \item {既存のシミュレータを利用するグループ}
  \item {\simname を利用するグループ}
\end{itemize}
の3つに分け、各グループに共通の pre-test を課す。その後、同じ分野について同等の内容を教え、post-test を課す。各グループで normalized gain がどのような値を示すかによって、\simname の教育効果を評価することができる。

また、pre-test と post-test として2種類の問題を提案する。

\subsubsection*{文字式をベースとした択一問題}
\ref{intro}で述べたように、\simname を用いることで文字式に対するある種の直感的な理解を与えることができると考えている。\simname を用いないグループと比べて択一問題の正答率や回答速度が上がれば、選択肢がどのような現象を表しているかを直感的に理解できているということである。

\subsubsection*{文字式をベースとした通常の記述式問題}
一般的な大学入試と同様の記述式問題のスコアも有意であると考える。さらに、このテストにおいてはケアレスミスの比率を考察することにも価値がある。\simname を用いるグループのケアレスミスが他のグループと比較して有意に少なければ、\simname によって自分が計算した文字式が妥当か判断する力がついたということである。

\subsubsection*{}


\bibliographystyle{junsrt}
\bibliography{thesis}

\end{document}