\documentclass[11pt, a4paper, oneside, twocolumn]{jsarticle}
\usepackage{url}
\usepackage[dvipdfmx]{graphicx}
\usepackage{geometry}
\geometry{top=1cm,  bottom=2cm, left=1.5cm, right=1.5cm}
\pagestyle{empty}
\usepackage{amsmath}

\newcommand{\simname}{SimSym}
\newcommand{\simnamealt}{Simulation with Symbols}

\title{物理学の学習のためのプログラマブルな
\\シミュレータと環境の提案}

\author{東京工業大学 情報理工学院 数理・計算科学系\\18B04657 木内康介\\指導教員 増原英彦教授}

\date{}

\begin{document}
\maketitle

\section{はじめに} \label{intro}
高等学校における物理学の授業において、生徒による実験は重要である。Holubova~\cite{holubova_2019}は、実験室での作業は理論的な概念を検証する最も重要な方法であり、生徒は実験を通して自分の考えを判断することができると述べている。しかし実際は、実験の実施は完全ではない。林らの調査~\cite{2015KJ00010038066}によると、力学分野における最も基本的な「運動の法則」に関する実験の経験は、2014年の調査時点で60\%程度にしか満たない。
% また、非常によく取り上げられる題材である「モンキーハンティング」に関しては10\%に満たない。考えられる理由としては、モンキーハンティングを実験するためには大掛かりな装置と空間が必要になることや、測定が難しいことなどがある。
理由としては、実験用の装置の準備や測定が難しいことや、実験を行うのに時間を要することが考えられる。

そこで実験の代替として考えられるのが、シミュレーションの利用である。シミュレーションを用いることで、誰でも実験と同様な学習効果を得ることができる。実際 Ajredini~\cite{ajredini_real_2014}は、実際の実験とシミュレーションで得られる知識に大きな差は無いと結論づけている。

しかし、既存のシミュレータでは実際に生徒が解く問題との間のギャップが大きい。一般的なシミュレータは、図~\ref{numeral_based}のようにそれぞれの値や計算結果を数値として表現している。このような
% 表現手法を数値ベースと呼ぶことにする。数値ベースの
シミュレータでは、生徒はバックグラウンドでどのような計算がされているのかを確認できず、結果の数値が確認できるだけである。一方、生徒が解くことになる物理の問題は図~\ref{symbol_based}のような文字式ベースの計算が主流である。そのため、自分が計算した結果の文字式とシミュレーションの結果が一致しているかを確認するためにシミュレーションで利用した値を代入して計算する必要がある。

\begin{figure}[hbt]
\centering
\includegraphics*[width=0.9\linewidth]{figure/PhET_example.png}
\caption{数値ベースのシミュレータ例 (PhET~\cite{perkins_phet_2006})} \label{numeral_based}
\end{figure}

\begin{figure}[hbt]
\begin{align}
  \left\{
  \begin{aligned}
    ma &= F & (1)\\
    v &= v_0 + at & (2)
  \end{aligned}
  \right. \nonumber
\end{align}
\begin{align}
  \text{(1)より,}\quad a &= \frac{F}{m} \nonumber\\
  \text{(2)に代入して,}\quad v &= v_0 + \dfrac{F}{m}t \nonumber
\end{align}
\caption{文字式ベースの計算例} \label{symbol_based}
\end{figure}

そこで本研究では、物体の動きを生徒自身が文字式で定義できるシミュレータである「\simname (\simnamealt)」を提案する。\simname では、生徒が定義した文字式に従って数値計算がなされ、シミュレーションが実行される。これにより、生徒が導出した文字式がどのような動きと対応するか簡単に確認することができる。

\section{\simname}
\subsection*{動きの定義}
\simname では図~\ref{simsym_fig1}のように物体の動きを文字式で定義できる。シミュレーションを実行するためには数値が必要なので、定数の値を設定する必要がある。また、時刻を表す変数 $t$ を自由に使用できる。

\begin{figure}[htb]
  \centering
  \includegraphics*[width=0.9\linewidth]{work/slide_img1-crop.pdf}
  \caption{\simname 上で物体の動きを文字式で定義する例} \label{simsym_fig1}
\end{figure}

\subsection*{シミュレーションの実行}
時刻 $t$ を変化させながら物体の位置を計算し描画することで、シミュレーションが実行できる。

\section{実装}
実装は、フロントエンドに lively.next\footnote{\url{https://lively-next.org}} を、文字式の計算に SymPy~\cite{meurer_sympy_2017} を用いた。SymPy は Python のライブラリであるが、WebAssembly で実装された CPython 処理系の Pyodide\footnote{\url{https://pyodide.org/en/stable/}} を用いることでブラウザ上で完結させた。また、数式の表示には KaTeX\footnote{\url{https://katex.org}} を用いた。

\subsection*{lively.next}
lively.next は Lively Kernel~\cite{ingalls_2008}から派生したプロジェクトで、Webプログラミング環境である。JavaScriptで記述されたコンポーネントをブラウザ上で組み合わせることで、GUIアプリケーションを作成することができる。\simname のフロントエンドは lively.next 上で作成した。

\subsection*{SymPy}
SymPy は、文字式の計算を可能にする Python ライブラリである。文字式の定義や数値を代入しての計算などを SymPy で行っている。Python のライブラリであるが、Pyodide を用いることで lively.next から直接扱うことが可能になる。

\begin{figure}[hbt]
\centering
\includegraphics*[width=0.9\linewidth]{work/slide_img2-crop.pdf}
\caption{\simname の実装の概要}
\end{figure}

\section{まとめと課題}
本研究では文字式を基にしたシミュレーションが可能であるシミュレータ「\simname」を提案した。\simname を用いることで、生徒は自身が導出した文字式がどのような動きと対応するか簡単に確認することができる。結果として、物理学に対するより直感的・本質的な理解が促進できると考える。

今後の課題として、\simname の教育効果の評価がある。評価手法として、Hake~\cite{hake_1998}が導入した normalized gain を用いる方法を提案する。満点を100とする pre-test と post-test の平均点をそれぞれ $\langle \textrm{pre\\-test} \rangle$, $\langle \textrm{post\\-test} \rangle$ とすると、normalized gain $\langle g \rangle$ は次のように定義される:
$$ \langle g \rangle = \dfrac{\langle \textrm{post\\-test} \rangle - \langle \textrm{pre\\-test} \rangle}{100 - \langle \textrm{pre\\-test} \rangle} $$
高校生数十人を、
\begin{itemize}
  \item {シミュレータを利用しないグループ}
  \item {既存のシミュレータを利用するグループ}
  \item {\simname を利用するグループ}
\end{itemize}
の3つに分け、各グループに共通の pre-test を課す。その後、同じ分野について同等の内容を教え、post-test を課す。各グループで normalized gain がどのような値を示すかによって、\simname の教育効果を評価することができると考える。

% \section{評価手法の提案}
% \simname の教育効果の評価については、Hake~\cite{hake_1998}が導入した normalized gain を用いる方法を提案する。満点を100とする pre-test と post-test の平均点をそれぞれ $\langle \textrm{pre\\-test} \rangle$, $\langle \textrm{post\\-test} \rangle$ とすると、normalized gain $\langle g \rangle$ は次のように定義される:
% $$ \langle g \rangle = \dfrac{\langle \textrm{post\\-test} \rangle - \langle \textrm{pre\\-test} \rangle}{100 - \langle \textrm{pre\\-test} \rangle} $$
% 例えば、平均の得点率が $20\%$ から $60\%$ になった場合と $60\%$ から $80\%$ になった場合では、どちらも $\langle g \rangle = 0.5$ となる。

% 具体的には、高校生数十人を対象とした調査を行う。対象を
% \begin{itemize}
%   \item {シミュレータを利用しないグループ}
%   \item {既存のシミュレータを利用するグループ}
%   \item {\simname を利用するグループ}
% \end{itemize}
% の3つに分け、各グループに共通の pre-test を課す。その後、同じ分野について同等の内容を教え、post-test を課す。各グループで normalized gain がどのような値を示すかによって、\simname の教育効果を評価することができる。

% また、pre-test と post-test として2種類の問題を提案する。

% \subsubsection*{文字式をベースとした択一問題}
% \ref{intro}で述べたように、\simname を用いることで文字式に対するある種の直感的な理解を与えることができると考えている。\simname を用いないグループと比べて択一問題の正答率や回答速度が上がれば、選択肢がどのような現象を表しているかを直感的に理解できているということである。

% \subsubsection*{文字式をベースとした通常の記述式問題}
% 一般的な大学入試と同様の記述式問題のスコアも有意であると考える。さらに、このテストにおいてはケアレスミスの比率を考察することにも価値がある。\simname を用いるグループのケアレスミスが他のグループと比較して有意に少なければ、\simname によって自分が計算した文字式が妥当か判断する力がついたということである。


\bibliographystyle{junsrt}
\bibliography{thesis}

\end{document}