\documentclass[11pt, a4paper, oneside, twocolumn]{jsarticle}
\usepackage{url}
\usepackage[dvipdfmx]{graphicx}
\usepackage{geometry}
\geometry{top=1cm,  bottom=2cm, left=1.5cm, right=1.5cm}
\pagestyle{empty}

\newcommand{\simname}{[TODO: ここにシミュレータの名前を入力]}

\title{物理学の学習のためのプログラマブルな
\\シミュレータと環境の提案}

\author{東京工業大学 情報理工学院 数理・計算科学系\\18B04657 木内康介\\指導教員 増原英彦教授}

\date{}

\begin{document}
\maketitle

\section{はじめに}
高等学校における物理学の授業において、生徒による実験は必要不可欠である。しかし実際は、実験の実施は完全ではない。林らの調査~\cite{2015KJ00010038066}によると、力学分野における最も基本的な「運動の法則」に関する実験の経験は、2014年の調査時点で60\%程度にしか満たない。また、非常によく取り上げられる題材である「モンキーハンティング」に関しては10\%に満たない。考えられる理由としては、モンキーハンティングを実験するためには大掛かりな装置と空間が必要になることや、測定が難しいことなどがある。

そこで代替として考えられるのが、シミュレーションの利用である。簡単に実験ができない内容であったとしても、シミュレーションを用いれば誰でも実験と同様な学習効果を得ることができる。実際 Ajredini~\cite{ajredini_real_2014}は、実際の実験とシミュレーションで得られる知識に大きな差は無いと結論づけている。

しかし、既存のシミュレータでは不十分な点も存在する。一般的な物理学の学習用のシミュレータでは、設定されたシチュエーションにおける物体の挙動を観測することはできるが、シチュエーションそのものを大きく変化させることはできない。即ち、「この物体の座標を変更するとどう動くか」「この物体の初速度を変更するとどう動くか」「重力加速度の値を変更するとどう動くか」などといった疑問全てに対するシミュレーションは提供できない。更に、シミュレーションはその特性上数値計算をベースに実行される。一方、大学入試などにおける物理の問題は文字式の計算をベースにしている。そのため、「自分が計算した結果のこの文字式は正しく物理現象を表しているのか」ということを既存のシミュレータで確認することは難しい。

そこで、シチュエーションを自分で設定でき、文字式をベースとしたシミュレーションが可能なシミュレータである「\simname」を提案する。

\bibliographystyle{junsrt}
\bibliography{thesis}

\end{document}