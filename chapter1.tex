\chapter{はじめに} \label{first}
%\pagenumbering{arabic}  %% ページ番号をアラビア数字に変更

高等学校における物理学の授業において、生徒による実験は必要不可欠である。文部科学省が平成30年に告示した高等学校学習指導要領の理科編には以下ように記されている:
\begin{quote}
探究的な学習は教育課程全体を通じて充実を図るべきものであるが,観察・実験等を重視して学習を行う教科である理科がその中核となって探究的な学習の充実を図っていくことが重要である。
\end{quote}
また、[TODO: 実験による学習効果を調査した論文を調べる]

しかし実際は、実験の実施は完全ではない。林らの調査~\cite{2015KJ00010038066}によると、力学分野における最も基本的な「運動の法則」に関する実験の経験は、2014年の調査時点で60\%程度にしか満たない。また、非常によく取り上げられる題材である「モンキーハンティング(\ref{モンキーハンティング})」に関しては10\%に満たない。考えられる理由としては、モンキーハンティングを実験するためには大掛かりな装置と空間が必要になることや、測定が難しいことなどがある。

そこで代替として考えられるのが、シミュレーションの利用である。
% 現在日本ではGIGAスクール構想に基づき、生徒に対して1人1台ずつのコンピュータ端末の配布が進められている。そのため、今まで以上にコンピュータを利用したシミュレーションを活用することが可能である。
簡単に実験ができない内容であったとしても、シミュレーションを用いれば誰でも実験と同様な学習効果を得ることができる。実際 Ajredini~\cite{ajredini_real_2014}は、実際の実験とシミュレーションで得られる知識に大きな差は無いと結論づけている。また、シミュレーションでは実験器具の準備などの作業に割く時間を削減することができ、思考・分析・議論により多くの時間を割くことができるとも述べている。

しかし、既存のシミュレータでは不十分な点も存在する。一般的な物理学の学習用のシミュレータでは、設定されたシチュエーションにおける物体の挙動を観測することはできるが、シチュエーションそのものを大きく変化させることはできない。即ち、「この物体の座標を変更するとどう動くか」「この物体の初速度を変更するとどう動くか」「重力加速度の値を変更するとどう動くか」などといった疑問全てに対するシミュレーションは提供できない。更に、シミュレーションはその特性上数値計算をベースに実行される。一方、大学入試などにおける物理の問題は文字式の計算をベースにしている。そのため、自分が計算した結果の文字式と自分が想定している動きが一致しているかを確認することができない。

そこで、本論文ではシチュエーションを自分で設定でき、文字式をベースとしたシミュレーションが可能なシミュレータである「\simname」を提案する。文字式をそのまま再現することができるシミュレータを用いれば、自分の計算結果を「実行」し、自分が想定している動きを正しく表せているかを確認できる。それにより、文字式と動きの対応を学習することができ、文字式に対するある種の直感的な理解を与えることができる。結果として、自分が計算した文字式が妥当かを判断したり、選択肢中に存在するありえない択を排除したりすることができるようになると考える。

本論文の構成は以下の通りである。
第\ref{related}章で、既存のシミュレータとそれを用いた実例について紹介する。
第\ref{method}章で、\simname の紹介とその効果を説明する。
第\ref{implementation}章で、\simname の実現方法を説明する。
第\ref{evaluation}章で、\simname の評価方法を提案する。
第\ref{conclusion}章で、まとめと今後の展望について述べる。
