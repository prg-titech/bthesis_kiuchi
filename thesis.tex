%
% Thesis Style 用サンプル TeX
%
% 注意:目次等を作成するために何度か platex にかけること
%
% 両面印刷する時は twoside にする
%
% 12pt にするのは最終手段
%
\documentclass[11pt,oneside,dvipdfmx]{jbook}
\usepackage{csg-thesis}
\usepackage[dvipdfmx]{graphicx}
\usepackage{url}
\usepackage{amsmath}
\usepackage{pdfpages}
\usepackage{float}
\usepackage[a4paper]{geometry}
\usepackage{booktabs}
\usepackage{multirow}

\usepackage{listings,jvlisting}
\lstset{
  basicstyle={\ttfamily},
  identifierstyle={\small},
  commentstyle={\smallitshape},
  keywordstyle={\small\bfseries},
  ndkeywordstyle={\small},
  stringstyle={\small\ttfamily},
  frame={tb},
  breaklines=true,
  columns=[l]{fullflexible},
  numbers=left,
  xrightmargin=0zw,
  xleftmargin=3zw,
  numberstyle={\scriptsize},
  stepnumber=1,
  numbersep=1zw,
  lineskip=-0.5ex
}

\newcommand{\simname}{SimSym}
\newcommand{\simnamealt}{Simulation with Symbols}

\usepackage[dvipdfmx]{hyperref}
\usepackage{pxjahyper}
\hypersetup{%
  setpagesize=false,%
  hidelinks,%
  colorlinks=false,%
  pdftitle={},%
  pdfsubject={},%
  pdfauthor={},%
  pdfkeywords={}%
}

\begin{document}

% 題目
% 適当に\\で区切って見やすくする
\title{%
物理学の学習のためのプログラマブルな
シミュレータと環境の提案
}

% 学位(学部と大学院で変更)
\degree{学士}
%\degree{修士}

% 名前
\author{木内 康介}

% 提出日
\date{令和5年2月6日}

% 卒業年度
\schoolyear{令和4年}

% 所属(学部と大学院で変更)
\department{東京工業大学 情報理工学院 数理・計算科学系}

% 学籍番号
\stnumber{18B04657}

% 指導教員(教官ではなくなった...)
\supervisor{増原 英彦 教授}

\maketitle

%%%%%%%%%%%%%%%%%%%%%%%%%%%%%%%%%%%%%%%%%%%%%%%%%%%%%%%%%%%%%%%%%%%%%%

% 概要
\begin{abstract}
高等学校での物理学の授業では、実験は重要である。しかし実際は、生徒全員が実験を経験しているわけではない~\cite{2015KJ00010038066}。理由としては、実験用の装置の準備や測定が難しいことや、実験を行うのに時間を要することが考えられる。そこで実験の代替として近年利用されているのが、物理実験のシミュレータである。シミュレータを用いることで、実験と同様の学習効果を得ることができる~\cite{ajredini_real_2014}。

しかし、既存のシミュレータでは不十分な点が存在する。既存のシミュレータは物理法則から現象を数式で記述し、シミュレーションに変換するまでを教育者が行なっている。学習者はシミュレーションを見ることで現実の物理現象を確認することができるが、それと物理法則との関係性は授業などで学ぶのみである。

そこで本研究では、学習者に物理系を定義させるシミュレータである \simname~(\simnamealt) を提案する。学習者は \simname で系内の物体をそのパラメータとともに定義し、その物体の運動を表す方程式を立式する。この際学習者は、次元の異なる物理量の和が存在する不正な方程式を立式すると警告されるなど、正しい物理系を作成するための補助を受ける。シミュレーションを実行すると、定義した物理系に基づいて数値計算がなされ、物体の運動が可視化される。さらに、 \simname には現実の運動を正しく表現した動作例が存在し、学習者が参考にすることができる。

\simname では、物理法則から現象を数式で記述し、シミュレーションに変換するまでの工程も学習者が行う。これにより、従来は授業などで教わる必要があった現実の物理現象と物理法則の間の対応を学習者自らの経験を通して理解することができる。

\end{abstract}

% 謝辞
\begin{acknowledgments}
本研究を進めるにあたり、増原英彦教授、叢悠悠助教にアドバイスやご指導をいただきました。この場を借りて感謝申し上げます。また、増原研究室の皆様に多くのコメントを頂きました。特に、ご自身の研究でお忙しい中で何度も協力してくださった田辺さん、私と同じく教育に関する研究をしており相談に乗ってくれた角田さん、学生室で雑談相手になってくれた津山くんに深く感謝しております。

% また、 lively.next 開発者の Robin, Linus, Jens にも助けていただきました。
% 本論文は以上の方々のご支援がなければ存在しえませんでした。

\end{acknowledgments}

%%%%%%%%%%%%%%%%%%%%%%%%%%%%%%%%%%%%%%%%%%%%%%%%%%%%%%%%%%%%%%%%%%%%%%

\tableofcontents       %% 目次

%
% 目次等にはローマ数字を使い、本文開始ページを 1 ページ目にできる
% この方が見た目がきれいであるが、全体のページ数は減って見える
% ここでローマ数字に変えた場合は chapter 1 でアラビア数字に戻すこと
%
\pagenumbering{roman}  %% ページ番号をローマ数字にする

% \listoffigures         %% 図目次(図がない場合は不要)
% \listoftables          %% 表目次(表がない場合は不要)

%%%%%%%%%%%%%%%%%%%%%%%%%%%%%%%%%%%%%%%%%%%%%%%%%%%%%%%%%%%%%%%%%%%%%%

\cleardoublepage
\pagenumbering{arabic}  %% ページ番号をローマ数字にする
%
% 本文
%
% 各章を各ファイルに書く
%
\chapter{はじめに} \label{first}
%\pagenumbering{arabic}  %% ページ番号をアラビア数字に変更

書くこと

\begin{itemize}
\item {
  物理教育におけるシミュレーションの位置付け
}
\item {
  物理教育でのシミュレーションの実例
}
\item {
  現在のシミュレーションでは自分の目的に適していないことの説明
}
\end{itemize}

本論文の構成は以下の通りである。第\ref{related}章で

%%%%
\chapter{関連研究} \label{related}

\section{PhET}

PhET\cite{Perkins2006PhETIS}\cite{PhET}について

PhETを用いた実例の紹介
\cite{prima_learning_2018}
\cite{rehman_teaching_2021}

\section{Scratch}

Scratch\cite{Scratch}について

Scratchを用いた実例の紹介
\cite{Lpez2015ScratchAA}


%%%%
\chapter{提案する内容} \label{method}

\begin{itemize}
\item {
  シミュレータの概要
}
\item {
  既存のシミュレータとの差異
}
\end{itemize}

%%%%
\chapter{実装}

Lively.next\cite{Lively.next}
$\Leftrightarrow$
Pyodide\cite{Pyodide}
$\Leftrightarrow$
Sympy\cite{SymPy}



%%%%
\chapter{評価手法}

\cite{Pucholt_2021}の手法を参考にする



%%%%
\chapter{まとめと展望} \label{conclusion}


%%%%%%%%%%%%%%%%%%%%%%%%%%%%%%%%%%%%%%%%%%%%%%%%%%%%%%%%%%%%%%%%%%%%%%

%
% 参考文献は直接書いてもいいが、bibtex を使うと便利
%  (1)このサンプルを platex にかける
%  (2)jbibtex thesis
%  (3)さらに platex にかける
%  (4)もう何回か platex にかける
%
% \bibliographystyle{csg-thesis}
\bibliographystyle{junsrt}
\bibliography{thesis}  %% thesis.bib というファイルを用意

%%%%%%%%%%%%%%%%%%%%%%%%%%%%%%%%%%%%%%%%%%%%%%%%%%%%%%%%%%%%%%%%%%%%%%

% 付録が必要ならつける
\appendix
\chapter{補足}

\section{モンキーハンティング} \label{モンキーハンティング}

以下のような形式の問題を総称してモンキーハンティングと呼ぶ。

\begin{quote}
小球を、位置 $(0, 0)$ から初速 $v_0$、 仰角 $\theta$ で発射したところ、位置 $(l, h)$ から自由落下してくる物体に衝突した。 $\tan \theta$ の満たすべき条件を求めよ。ただし、重力加速度の大きさを $g$ とする。
\end{quote}

\begin{figure}[htb]
\centering
\includegraphics{work/monkey_hunting.png}
\end{figure}


\end{document}
