%%%%
\chapter{関連研究} \label{related}

\section{PhET}

PhET(Physics Education Technology)\cite{Perkins2006PhETIS}は、コロラド大学ボルダー校によるプロジェクトで、物理学の教育に活用できるシミュレーションの作成を目標としている。2023年1月現在、ウェブサイト\cite{PhET}上では50以上のシミュレーションが公開されている。また、
物理学のみならず化学・数学・生物学・地球科学などのシミュレーションも公開されている。

ここからは、PhETの物理学シミュレーションに的を絞ってより詳しく紹介する。PhETでは、学習レベルに応じたシミュレーションの分類も行っている。各レベル毎のシミュレーションの数は表\ref{PhET_sim_count_table}の通りである。

\begin{table}[htb]
\label{PhET_sim_count_table}
\centering
\caption{PhET における GRADE LEVEL 毎のシミュレーション数}
\begin{tabular}{cr}
  GRADE LEVEL & 個数 \\
  \hline
  Elementary School & 23 \\
  Middle School & 37 \\
  High School & 49 \\
  University & 47 \\
\end{tabular}
\end{table}

\subsection{PhETを用いた実例の紹介}

Prima\cite{prima_learning_2018}は、インドネシアの中学校の生徒にPhETを用いて太陽系について教える実験を行った。Primaは、PhETを利用する効果をN-Gain(normalized gain)を用いて評価している。満点を $100$ とする pre-test と post-test の平均点をそれぞれ $\langle \textrm{pre\\-test} \rangle$, $\langle \textrm{post\\-test} \rangle$ とすると、 N-Gain $\langle g \rangle$ は以下のように求められる:
$$ \langle g \rangle = \dfrac{\langle \textrm{post\\-test} \rangle - \langle \textrm{pre\\-test} \rangle}{100 - \langle \textrm{pre\\-test} \rangle} $$
また、テストは Bloom's Taxonomy に基づき Remembering, Understanding, Applying, Analyzing の4領域で行われた。

\textgt{この調子で書いていくと\cite{prima_learning_2018}の内容を翻訳するだけになるので一旦中断}

\cite{rehman_teaching_2021}

\section{Scratch}

Scratch\cite{Scratch}について

Scratchを用いた実例の紹介
\cite{Lpez2015ScratchAA}

