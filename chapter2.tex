%%%%
\chapter{関連研究} \label{related}

この章では、既存のシミュレータとそれを用いた教育の実例について紹介する。

\section{PhET}

PhET(Physics Education Technology)~\cite{Perkins2006PhETIS}は、コロラド大学ボルダー校によるプロジェクトで、物理学の教育に活用できるシミュレーションの作成を目標としている。2023年1月現在、ウェブサイト~\cite{PhET}上では50以上のシミュレーションが公開されている。また、
物理学のみならず化学・数学・生物学・地球科学などのシミュレーションも公開されている。

% ここからは、PhETの物理学シミュレーションに的を絞ってより詳しく紹介する。PhETでは、学習レベルに応じたシミュレーションの分類も行っている。各レベル毎のシミュレーションの数は表\ref{PhET_sim_count_table}の通りである。

% \begin{table}[htb]
% \centering
% \caption{PhET における GRADE LEVEL 毎のシミュレーション数} \label{PhET_sim_count_table}
% \begin{tabular}{cr}
%   GRADE LEVEL & 個数 \\
%   \hline
%   Elementary School & 23 \\
%   Middle School & 37 \\
%   High School & 49 \\
%   University & 47 \\
% \end{tabular}
% \end{table}

\subsection{PhETを用いた実例}

Prima~\cite{prima_learning_2018}は、インドネシアの中学校の生徒にPhETを用いて太陽系について教える実験を行った。Primaは、PhETを利用する効果をN-Gain(normalized gain)を用いて評価している。満点を $100$ とする pre-test と post-test の平均点をそれぞれ $\langle \textrm{pre\\-test} \rangle$, $\langle \textrm{post\\-test} \rangle$ とすると、 N-Gain $\langle g \rangle$ は以下のように求められる:
$$ \langle g \rangle = \dfrac{\langle \textrm{post\\-test} \rangle - \langle \textrm{pre\\-test} \rangle}{100 - \langle \textrm{pre\\-test} \rangle} $$
また、テストは Bloom's Taxonomy に基づき Remembering, Understanding, Applying, Analyzing の4領域で行われた。結果は表\ref{prima_table}の通りである。

\begin{table}[h]
\centering
\begin{tabular}{lcccccc}
  \toprule
  \multirow{2}{*}{Cognitive level} & \multicolumn{3}{c}{Control Group} & \multicolumn{3}{c}{Experiment Group}\\
  \cmidrule(lr){2-4} \cmidrule(lr){5-7}
  & pretest & posttest & N-Gain & pretest & posttest & N-Gain\\
  \midrule
  Remembering (C1) & 33.33 & 76.19 & 0.64 & 44.44 & 84.12 & 0.71\\
  Understanding (C2) & 38.62 & 57.67 & 0.31 & 51.85 & 75.66 & 0.49\\
  Applying (C3) & 54.76 & 80.95 & 0.57 & 69.04 & 92.85 & 0.76\\
  Analyzing (C4) & 56.34 & 75.39 & 0.43 & 57.14 & 80.15 & 0.53\\
  \bottomrule
\end{tabular}

\caption{Primaによる実験の結果}
\label{prima_table}
\end{table}

% Cognitive level Control Group Experiment Group pretest posttest N-Gain pretest posttest N-Gain Remembering (C1) 33.33 76.19 0.64 44.44 84.12 0.71 Understanding (C2) 38.62 57.67 0.31 51.85 75.66 0.49 Applying (C3) 54.76 80.95 0.57 69.04 92.85 0.76 Analyzing (C4) 56.34 75.39 0.43 57.14 80.15 0.53

\textgt{この調子で書いていくと~\cite{prima_learning_2018}の内容を翻訳するだけになるので一旦中断}



~\cite{rehman_teaching_2021}

\section{Scratch}

Scratch~\cite{Scratch}について

Scratchを用いた実例の紹介
~\cite{Lpez2015ScratchAA}

