\chapter{関連研究} \label{related}

この章では、既存のシミュレータとそれを用いた教育の実例について紹介する。

\section{PhET}

PhET(Physics Education Technology)~\cite{perkins_phet_2006} は、コロラド大学ボルダー校によるプロジェクトで、物理学の教育に活用できるシミュレーションの作成を目標としている。2023年2月現在、ウェブサイト\footnote{\url{https://phet.colorado.edu}}上では50以上のシミュレーションが公開されている。また、
物理学のみならず化学・数学・生物学・地球科学などのシミュレーションも公開されている。

\subsection{PhETの教育効果}

\subsubsection{Ajrediniの 実験}

Ajredini~\cite{ajredini_real_2014} は、マケドニアの高校生に PhET を用いて電荷について教える実験を行った。実際の実験を行うグループ、PhET のシミュレーションを利用するグループ、実験を行わないグループの3つに分け、電荷に関する授業を行った。その後、いくつかのシチュエーションを指定し、スケッチを描かせた。各シチュエーションと正答率は表~\ref{ajredini_result}の通りであり、実際の実験と PhET との間の有意差は見られなかった。

\begin{table}[htb]
\noindent\rule{\linewidth}{0.4pt}
\begin{quote}
2つの軽い中性の金属球を糸でぶら下げる。次のような場合についてスケッチを描け。
\end{quote}
\begin{enumerate}
\renewcommand{\labelenumi}{\alph{enumi}).}
\item 両方の球体に毛布で擦ったプラスチック棒を接触させ、帯電させる
\item (a) の状態で、球体をより遠くに置く
\item 球体 A は毛布で擦ったプラスチック棒で帯電させ、球体 B は皮で擦ったガラス棒で帯電させる。
\item 両方の球体をプラスチック棒で帯電させるが、球体 A は球体 B より多く帯電させる。
\item 球体 A はプラスチック棒で帯電させ、球体 B は中性のままにする
\item 球体 A はガラス棒で帯電させ、球体 B は中性のままにする
\end{enumerate}
\centering
\begin{tabular}{rrrr}
  \toprule
  グループ & 実験 & PhET & 座学 \\
  \midrule
  a) & 75 & 66 & 43 \\
  b) & 35 & 34 & 14 \\
  c) & 67 & 70 & 31 \\
  d) & 23 & 26 & 12 \\
  e) and f) & 11 & 7 & 3 \\
  \bottomrule
\end{tabular}
\caption{Ajredini による実験の結果} \label{ajredini_result}
\end{table}

\subsubsection{Prima の実験}

Prima~\cite{prima_learning_2018} は、インドネシアの中学生に PhET を用いて太陽系について教える実験を行った。PhET を用いるグループと用いないグループに分け授業を行い、その結果を Normalized Gain を用いて評価している。満点を $100$ とする pre-test と post-test の平均点をそれぞれ $\langle \textrm{pre\\-test} \rangle$, $\langle \textrm{post\\-test} \rangle$ とすると、 Normalized Gain $\langle g \rangle$ は以下のように求められる:
$$ \langle g \rangle = \dfrac{\langle \textrm{post\\-test} \rangle - \langle \textrm{pre\\-test} \rangle}{100 - \langle \textrm{pre\\-test} \rangle} $$
また、テストは Bloom's Taxonomy に基づき Remembering, Understanding, Applying, Analyzing の4領域で行われた。結果は表\ref{prima_table}の通りであり、全ての分野において PhET を用いた方が Normalized Gain が大きい。

\begin{table}[htb]
\centering
\begin{tabular}{lcccccc}
  \toprule
  \multirow{2}{*}{Cognitive level} & \multicolumn{3}{c}{Control Group} & \multicolumn{3}{c}{Experiment Group}\\
  \cmidrule(lr){2-4} \cmidrule(lr){5-7}
  & pretest & posttest & N-Gain & pretest & posttest & N-Gain\\
  \midrule
  Remembering (C1) & 33.33 & 76.19 & 0.64 & 44.44 & 84.12 & 0.71\\
  Understanding (C2) & 38.62 & 57.67 & 0.31 & 51.85 & 75.66 & 0.49\\
  Applying (C3) & 54.76 & 80.95 & 0.57 & 69.04 & 92.85 & 0.76\\
  Analyzing (C4) & 56.34 & 75.39 & 0.43 & 57.14 & 80.15 & 0.53\\
  \bottomrule
\end{tabular}

\caption{Primaによる実験の結果}
\label{prima_table}
\end{table}

\clearpage
\subsection*{Rehman の実験}
Rehman~\cite{rehman_teaching_2021} は、パキスタンの高校生に PhET を用いて重さと質量の概念について教える実験を行った。また、この授業は週5回・1ヶ月間行われた。結果は表~\ref{rehman_result}の通りであり、有意差が認められる。

\begin{table}[htb]
\centering
\begin{tabular}{lcccccc}
\toprule
& Control Group & Experiment Group \\
\midrule
pretest & 11.82 & 12.68 \\
posttest & 16.84 & 29.46 \\
\bottomrule
\end{tabular}
\caption{Rehman による実験の結果} \label{rehman_result}
\end{table}

\section{Scratch}

Scratch\footnote{\url{https://scratch.mit.edu}} について

Scratchを用いた実例の紹介
~\cite{Lpez2015ScratchAA}

